%documentclass[dvipdfm, a4paper, 12pt]{book} %book, article ou report
\usepackage[brazil]{babel} %linguagem do documento
\usepackage[utf8]{inputenc} %reconhece acento e cedilha
\usepackage{amssymb, amsmath, pxfonts} %permite simbolos matemáticos
\usepackage{cases}
\usepackage{mathrsfs} %permite uso de fontes para conjuntos
\usepackage[normalem]{ulem} %permite sublinhar palavras
\usepackage{mathrsfs} %permite o uso de letras trabalhadas
\usepackage[
	top=1cm,
	left=1cm,
	right=1cm,
	bottom=1.5cm
	]{geometry} %margens
\usepackage{graphicx} %permite inserir figuras
\usepackage[usenames]{color} %permite letras coloridas
\usepackage{makeidx} %pra criar índice remissivo

\usepackage[table, xcdraw]{xcolor}
\usepackage{longtable}
\usepackage{lipsum}
